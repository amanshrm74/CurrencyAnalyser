\chapter{Understanding Cryptocurrency}\label{understandingcryptocurrencych}

Since the first signs of digital finance in the 1970s, the financial services industry has relied more and more on new technologies and advancements in existing technologies. With the advent of the internet in the 1990s, becoming popular and more accessible in the 2000s, online banking was soon to become a commonplace financial service. As the internet grew and became faster, we witnessed an increase in both companies and individuals taking advantage of a variety of digital finance software, with respect to buying and selling goods and services and even trading stock.

\section{The Arrival of the First Digital Currency}\label{secfirstcrypto}

One of the first plausible instances of a digital currency, which paved the way for all cryptocurrencies we know today, was DigiCash. In 1983, David Chaum proposed the idea of using "blind signatures" for untraceable digital payments \cite{chaum83}. Similarly to the fears of the public today, Chaum discusses the issue of privacy in banking, also arguing that there would be a need to prevent digital payments from being used inappropriately such as in a criminal manner. Chaum follows by identifying the issue that knowledge of payment details such as payer or recipient details, by anyone other than the payer can often reveal sensitive information about that payer, such as their interests or whereabouts. Furthermore, Chaum also highlights the lack of security and control related to traditional bank notes and cheques. To resolve this issue, Chaum proposed the use of blind signatures, which work by concealing the content of a message through the use of "nested envelopes" (an envelope within an envelope) containing the message being passed back and forth between payer and recipient. More importantly, Chaum describes how this blind signatures system could be used in the implementation of an untraceable payments system, outlining an example of how a single transaction would take place - 
\begin{quote}
    The payer chooses a value to send to the recipient, and forwards the note to the bank. Thank signs the note, debits the payer's account, and returns the signed note to the payer. At this point the amount has been deducted from the payer's account, and they now have a note which is verified by the bank. The payer makes sure the value of the note is the same as what they initially sent to the bank, terminating the process if not. The payer then provides the note to the recipient, who checks the note. If the note is valid, the recipient forwards the note to the bank. The bank checks the validity of the note, terminating the process if any discrepancies exist. If the note is valid, the bank adds that note to a comprehensive list of cleared notes, terminating if the note is already on the list. Following success in all steps, the bank credits the recipient's account and informs them of acceptance \cite{chaum83}.
\end{quote}

This method of transferring money indicates that any transactions would be perfectly valid, as any fraudulent transactions would be made obvious at some stage, either through any irregularities in a transaction or by its details already existing in a list of transactions. Chaum went on to implement this system in his electronic cash business venture, DigiCash, formed in 1990 \cite{digicash}. Built on the idea of freeing digital currency from the control of any government, DigiCash and its underlying technology seemed promising at the time. However despite the viability of DigiCash on paper, the company ultimately failed due to a range of issues such as internal conflict and lack of funding \cite{digicrash}. One could also argue that DigiCash failed partly due to the internet still being relatively young, and therefore demand not yet existing for an online currency.

Times have since changed, and the unprecedented popularity of the internet from the early 21st century on has of course come with added concerns from its users regarding privacy. While DigiCash may have been before its time, the underlying concept of untraceable payments now seemed more desirable, and needed, than ever. 

\section{The Arrival of Modern Cryptocurrency}
The beginning of the rise of cryptocurrency can be pinpointed in 2009, when the the first major cryptocurrency was released to the public (\textit{see section \ref{btcsec}}). Much like traditional currency, any cryptocurrency is an asset, designed to be traded in exchange for goods and services. Cryptocurrency is based on cryptography, the study of breaking or creating of codes and ciphers to either encrypt plain text or decrypt cipher text, in order to keep the exchange of digital information safe and secure. Based on \textcolor{NavyBlue}{\href{www.coinmarketcap.com}{\textit{CoinMarketCap}}} figures, one of the most widely used websites for tracking the size and price of various cryptocurrencies, there are currently just over 1500 cryptocurrencies in existence today \cite{coinmarketcap}, with some of the most popular being Bitcoin, Ethereum, Litecoin and Ripple. When compared to the relatively small number of the 180 traditional currencies in circulation throughout the world, one might think that cryptocurrencies are more popular than traditional currencies. This is not the case however, mostly due to the fact that anyone can develop and release a cryptocurrency into the virtual world, as explained in \textit{\ref{secblockchaintech}: Blockchain Technology}.

\subsection{An Explanation of Modern Cryptocurrency}\label{expmodern}

Each cryptocurrency is its own self-contained system, separate from other cryptocurrencies, much like any traditional currency is its own entity. While the features of each individual currency may differ, such as their value or total available supply, their fundamental concepts are the same and undoubtedly very similar to Chaum's aspirations - an asset, free from the control of one government, party or person, and untraceable back to whomever traded the currency. The main features of most cryptocurrencies include decentralisation, anonymity and ability to be traded online through exchanges.

\subsubsection{Decentralised Systems}\label{decentralised}
The simple explanation for a decentralised system is to say it is owned and managed by the public, as opposed to one single centralised entity like a bank or government. While the system varies from currency to currency, most implement this idea through the use of time-stamps or unique serial numbers on each transaction. These unique transactions are recorded in a global ledger, a copy of which is kept on every currency owner's machine. Any anomalies in a transaction will be made obvious when checking the global ledger, avoiding the need for a trusted third party to verify transactions. This feature makes it virtually impossible to forge transactions into any account. The technology behind this global ledger is explained in greater detail in \textit{\ref{secblockchaintech}: Blockchain Technology}.

\subsubsection{Anonymity}\label{anoncrypto}
As mentioned previously, one of the main flaws with exchange of traditional currencies is the ability to identify sensitive information related to the sender, such as their location or interests. Possibly the most attractive feature of various cryptocurrencies is that they do not store any identifying data related to the user. An owner of a cryptocurrency keeps that currency in a digital "wallet", which is really just an encrypted address. If a person owns multiple currencies, they will have one address for each currency. Similarly, if a user owns a large quantity of one currency, they may want to distribute their currency across multiple addresses, in case one address is lost or has a breach in security. While this address is contained in the global ledger and available for every user to see, it is completely separate from a user's identity and completely the user's responsibility to control. For example, the address \textcolor{OliveGreen}{\mintinline{python}{3Pj8fcmuJeEoceD35f7vcgDNk2dtEGgmV9}} does not reveal any information about its owner, keeping their privacy intact.

Anonymity is extremely important in cryptocurrency in terms of privacy, but also largely in terms of keeping users with vast investments safe from being threatened in their non-virtual lives.

\subsubsection{Mining of Cryptocurrency}\label{mining}
For most cryptocurrencies, a process called "mining" must take place in order to confirm transactions and add them to the global ledger. This is not something every single user must do, but rather a process that can be done by anyone who wishes to and in exchange for a reward, often in the form of the currency they are mining. Cryptocurrency miners are given a very complex mathematical problem, and when the problem is solved the transaction is added to the ledger. These problems are kept at an extremely high level of difficulty and increase in difficulty to keep up with advancements in processing power, and are therefore mostly done by those with extremely powerful computers. The mining system ensures transactions are valid and adds to the overall security of cryptocurrencies.

\subsubsection{Trading Cryptocurrency}\label{tradingcrypto}
In order to trade cryptocurrency, one must have an address to store their currency in. There a number of methods and platforms for trading cryptocurrency, but most users start trading cryptocurrency by signing up to trading sites such as \textcolor{NavyBlue}{\href{https://www.coinbase.com/}{\textit{Coinbase}}} \cite{coinbase} or \textcolor{NavyBlue}{\href{https://www.gdax.com/}{\textit{GDAX}}} \cite{gdax}. Sites like these offer a user-friendly interface, simplifying the buying and selling of the process as much as possible. Once a user has signed up, verified their identification and added an associated bank account or card, they can buy and sell a variety of coins at the touch of a button and without worrying about any underlying keys or addresses. These sites also allow the transfer of currency from one account to another provided the destination address is known. These sites also provide up to date prices, often displayed as a graph over a period of time, and users can access their own key or balance at any time. One should keep in mind that these sites are not completely free and that there is a small fee to pay on every transaction, but these sites are an excellent way for beginners to get started in trading cryptocurrencies.

\subsection{Comparing Traditional Currency and Cryptocurrency}\label{sssectradvscrypto}
As outlined previously, there are many similarities and differences to be found between traditional currencies and cryptocurrencies.

Firstly, traditional currencies rely on centralised banks and governments to regulate and verify all transactions. This means that one person or group could destroy any proofs of transactions or change any account balances, simply by breaking into or hacking one location. In contrast, cryptocurrencies are decentralised, meaning any potential threat would have to hack in to every user's machine in order to destroy records or accounts. Of course, this would be an issue if there were little or no users but the growing number of owners of cryptocurrency means the likelihood of this is very small.

The decentralised nature of cryptocurrencies and their global ledger also adds to the credibility of each transaction. In contrast to the relative ease of which someone could forge a paper note, the global ledger means cryptocurrency transactions are virtually impossible to forge.

Additionally, any exchange of traditional currency carried out digitally (such as online banking or credit cards) can be easily traced back to the account and person it came from. Use of wallet addresses in cryptocurrency, as explained previously in \textit{\ref{anoncrypto}: Anonimity}, still means the source and destination of the transaction can be seen without exposing sensitive information related to any party.

\subsection{A Focus on Bitcoin}\label{btcsec}
The arrival of modern cryptocurrency can be pinpointed in 2009, when the first major cryptocurrency was released to the public. The concept of this first cryptocurrency, Bitcoin, was introduced in a paper published in 2008 by an author or authors under the supposed pseudonym of \textit{Satoshi Nakamoto}. In their paper, Nakamoto proposes the idea of a peer-to-peer electronic cash system, based on digital signatures and independent of any financial institution or government \cite{snakamoto}. Nakamoto also outlines a solution to verifying transactions, preventing "double-spending" using a proof-of-work method. This would involve time-stamping transactions, adding them  to a chain of hash-based proofs, which cannot change one transaction without redoing all previous entries in the proof-of-work. In order to change any entries in the chain, a single CPU would have to redo all entries before any new entries are added, which would require phenomenal computing power. Nakamoto also adds that this means the system is safe from threat as long as "honest" nodes collectively hold more CPU power than any group of attacking nodes.

\subsubsection{The Price of Bitcoin}
The first 50 coins were mined and released by Nakamoto in 2009, when the price per coin was estimated to have been a mere \$0.001 US dollars. Bitcoin can be traded on various exchanges, as described in \textit{\ref{tradingcrypto}: Trading Cryptocurrency}, but what is considered the first real world transaction of Bitcoin occurred in 2010, long before most of these exchanges existed. This transaction of Bitcoin occurred when programmer Laszlo Hanyecz posted on a Bitcoin Forum, offering 10,000 Bitcoin in exchange for someone to order a pizza to his home in Florida \cite{btcpizza}. The post still exists today, containing comments from various users at the time as well as more recent comments lightheartedly highlighting the rise in price of Bitcoin. 

The currency stayed at very low prices until late 2010, when the price rose to \$0.36 per coin. The price began rising in February 2011 and peaked at \$1.06, reaching "dollar parity", a much sought-after milestone \cite{BWrisefall}. The price of Bitcoin continued to rise, fuelled by media coverage, and spiked to over \$100 and  in 2013. Still rising, prices reached over \$900 at the end of 2013, eventually settling between \$400 - \$800 for most of 2014, 2015, and 2016 \cite{coindeskprices}. 2017 saw Bitcoin prices soar from just under \$1000 to a staggering \$17,500. One cannot help but be reminded of those two pizzas Hanyecz spent 10,000 Bitcoin on, now worth a staggering \$175 million dollars.

In 2018, the price of Bitcoin fell by roughly 60\% and has hovered consistently between \$5000 and \$7000 since. Even so, this is still exceptionally higher than any other cryptocurrency. Ethereum, the second most expensive coin, has averaged in the last month at around \$400-\$700 per coin, a mere 10\% of the price of a Bitcoin. 

\section{Blockchain Technology}\label{secblockchaintech}
While there are many different technologies behind various cryptocurrencies, arguably the most well known and important is the concept of blockchain. Blockchain, aforementioned briefly in \textit{\ref{expmodern}: An Explanation of Modern Cryptocurrency}, is the global ledger in which all transactions of a cryptocurrency are recorded. Blockchain technology was first proposed by Nakamoto in \cite{snakamoto}, and thus in this section we will be discussing blockchain technology in relation to Bitcoin.

As outlined by Nakamoto in \cite{snakamoto}, the blockchain consists of multiple blocks, added to the chain in a chronological order. Each block consists of a number of transactions, which are stored in the block as hashed items. In order for a new block to be added, it must be verified through the mining process. When a miner joins the Bitcoin network and downloads the software for validating and relaying transactions, a copy of the blockchain is also automatically downloaded \cite{blkchMS}. 

Validation of a block, discussed in \textit{\ref{mining}: Mining of Cryptocurrency}, involves solving a complex computational problem and requires immense processing power. Once a block has been validated, it is added to the chain and the miner is rewarded. Every block, containing information for every transaction, including the first "genesis" block is available to be viewed either on a local copy of the blockchain or on sites such as \textcolor{NavyBlue}{\href{https://blockchain.info/}{Blockchain.info}}. The Blockchain website in particular provides details for any given block, such as block number, number of transactions within the block, block time-stamp, and block reward. Most importantly, each block also stores the hashed key of the previous and next block in the chain, linking all blocks together.

\subsubsection{Block Rewards}

When Bitcoin was first released, the reward for validating a block was 50 Bitcoin, as pointed out by J. Donnelly in \cite{halving} with reference to Bitcoin's source code \cite{btcsc}. This meant that 50 new coins were released into the market. However, in order to counteract expected increase in demand, a halving function was implemented. There would be 64 "halvings", and a halving would occur every 210,000 blocks. The very first halving event occured in 2012 and the reward for a block reduced to 25 Bitcoin, followed by a second halving event in 2016, when the reward fell to 12.5 Bitcoin. The next halving event is expected to occur in 2020, and further events will occur after each 210,000 blocks until there have been 64 halvings, at which point the 21 million coin supply of Bitcoin will be reached \cite{halving} \cite{btcsc}.


\subsubsection{Trusting the Blockchain}

As previously highlighted in \textit{\ref{decentralised}: Decentralised Systems}, the blockchain is a global ledger of sorts, with identical copies of the chain stored on the machines of those who mine the relevant cryptocurrency. The decentralised nature of the blockchain means that no one person or group has control over the blockchain. In \cite{snakamoto}, Nakamoto explains that any attacking person or group would have to have more CPU power than all of those who were honest within the network in order to change any blocks existing in the chain. Furthermore, due to each block storing the hashed keys of both the previous and next block in the chain, the attacking node or nodes would have to change all previous entries in the chain before a new block was added and the chain updated.

While it should never be considered impossible to manipulate the blockchain, it would be highly difficult to. Perhaps when quantum computers emerge with considerably more processing power than that of today's computers, there could be a significant threat to the blockchain, but the processing capabilities of today's machines are certainly not a threat. 

\section{The Viability of Cryptocurrency}
Based on the fundamental ideas discussed in this chapter, cryptocurrencies would appear to be a very promising alternative to traditional currency when trading in exchange for goods and services. While traditional and digital currencies have their similarities, it is ultimately the innovative concepts brought to light by various cryptocurrencies which make them a promising alternative to traditional currencies.

Due to its global and decentralised nature, cryptocurrency can be used by anyone, anywhere, provided they have access to a digital device and a digital address from which to send and receive currency. When Bitcoin was first released in 2009, the global economy was in the depths of a severe recession and trust in financial institutions and governments was at an all time low. A certain unease still exists around how much trust should be placed in large financial bodies, but that trust is not needed in cryptocurrency. Technologies such as the blockchain ensure that no one person or group has control over the ledger, and not even a large group working together to corrupt the chain could be fast or powerful enough to succeed. Therefore the records remain intact, and can be trusted to be completely truthful. 

The security of cryptocurrency is undoubtedly an attractive feature to any potential investor, however the anonymity that comes with trading in cryptocurrency is something to be valued also. With recent privacy scandals such as the Facebook-Cambridge Analytica debacle \cite{camanalytica}, concern around public privacy is at its highest. While financial institutions are undoubtedly more secure than any social network, it would not be impossible for any person or group to compromise this security and obtain the records of account holders, containing addresses, transaction destinations, account balance and other sensitive identifying information. The features of cryptocurrency discussed in \textit{\ref{anoncrypto}: Anonymity} would reassure any user that their sensitive data is safe. By using a single hashed address of seemingly random numbers and letters for both sending and receiving a given amount, cryptocurrencies do not require any information similar to what financial institutions require. This system which requires absolutely no trust in any third party, is indeed a much more attractive approach than that of financial institutions and governments.

To conclude, in this chapter we have explained the fundamentals of what exactly cryptocurrency is, followed by an analysis of how cryptocurrencies work and their underlying technologies. Based mainly on its security features but also on its rising popularity, we can consider at least the largest cryptocurrencies such as Bitcoin and Ethererum to be a viable asset in terms of value and trading. While it is difficult to predict how the future of cryptocurrency will pan out or how the technology will endure in the long term, the solid foundations on which the concept is built suggest it will endure for some time.

Of course, this chapter has centred on the benefits associated with cryptocurrency. In the following chapter, we will discuss some of its shortcomings and flaws, with a focus on how volatile the cryptocurrency markets can be.
